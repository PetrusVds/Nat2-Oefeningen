\textbf{\underline{OZ 1 - Magnetische velden voor beginners - Oefening 1:}}
\vspace{0.5cm}

\textbf{Proton in magneetveld:} Een proton met een snelheid $ 4,00 \cdot 10^{6} $ m/s beweegt door een magneetveld van $ 1,70 T $ en ondergaat daardoor een kracht van $ 8,20 \cdot 10^{-13} N $. Wat is de hoek tussen het magnetische veld en de snelheidsvector van het proton?

\begin{description}[labelwidth=1.5cm, leftmargin=!]
    \item[Geg. :]   $ q = 1,60 \cdot 10^{-19} $ C; $ v = 4 \cdot 10^{6} $ m/s; $ B = 1,70 $ T; $ F = 8,20 \cdot 10^{-13} $ N;
    \item[Gevr. :]  $ \theta $;
    \item[Opl. :]   $ \vec{F} = q \vec{v} \times \vec{B} $
    
                    \hspace{-0.58cm} $ \Rightarrow
                    F = q v B \sin{\theta} $
    
                    \hspace{-0.58cm} $ \Leftrightarrow
                    \theta = \arcsin{\frac{F}{q v B}} $
                    
                    \hspace{0.12cm} $
                    = \arcsin{\frac{8,20 \cdot 10^{-13}}{1,60 \cdot 10^{-19} \cdot 4 \cdot 10^6 \cdot 1,70}} $
                    
                    \hspace{0.12cm} $
                    = 48,91^{\circ} \approx 48,9^{\circ} $
\end{description}

\vspace{1cm}
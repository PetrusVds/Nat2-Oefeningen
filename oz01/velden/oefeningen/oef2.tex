\textbf{\underline{OZ 1 - Magnetische velden voor beginners - Oefening 2:}}
\vspace{0.5cm}

\textbf{Magnetische kracht op een stroomvoerende geleider:} Beeld je in dat er een stroomvoerende draad met dichtheid $ 2,40 $ g/m rond de aarde is gespannen ter hoogte van de (magnetische) evenaar. Neem aan dat het aardmagnetische veld daar een constante grootte heeft van $ 28,0 \ \mu$T, parallel is  met het aardoppervlak en wijst naar het noorden. Welke stroom (grootte en richting) moet er door de draad gestuurd worden om er voor te zorgen dat de draad blijft leviteren net boven de grond?

\begin{description}[labelwidth=1.5cm, leftmargin=!]
    \item[Geg. :]   $ \frac{dm}{ds} = 2,40 $ g/m $ = 2,40 \cdot 10^{-3} $ kg/m; $ 28,0 \ \mu$T; $ \theta = 90^{\circ} $;
    \item[Gevr. :]  $ \vec{I} $;
    \item[Opl. :]   $ d\vec{F} = I d\vec{s} \times \vec{B} $
    
                    \hspace{-0.58cm} $ \Leftrightarrow
                    dm \cdot g = I \cdot ds \cdot B \cdot \sin{90^{\circ}} $
    
                    \hspace{-0.58cm} $ \Leftrightarrow
                    I = \frac{dm \cdot g}{ds \cdot B \cdot \sin{90^{\circ}}} $
    
                    \hspace{0.13cm} $
                    = \frac{2,40 \cdot 10^{-3} \cdot 9,81}{28 \cdot 10^{-6} \cdot \sin{90^{\circ}}} $
    
                    \hspace{0.13cm} $
                    = 840,85714 $ A $ \approx 841 $ A;
\end{description}

\vspace{1cm}
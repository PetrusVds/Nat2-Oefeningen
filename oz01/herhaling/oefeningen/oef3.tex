\textbf{\underline{OZ 1 - Herhaling - Oefening 3:}}
\vspace{0.5cm}

Een klein rigide object draagt een positieve lading en een negatieve lading van $3,50 $nC. De oriëntatie van het object is zo dat de positieve lading zich op ($-1,20 $mm,
$1,10 $mm) bevindt en de negatieve lading op ($1,40 $mm, $-1,30$ mm). Het object wordt geplaatst in
een elektrisch veld $\Vec{E} = (7800\hat{i} - 4900\hat{j})$ N/C.

\begin{enumerate}[(a)]
    \item Wat is het elektrisch dipoolmoment van het object?
    \item Wat is de torsie die het elektrisch veld op het object uitoefent?
    \item Wat is de potentiële energie van het object-veld systeem als het object in deze oriëntatie blijft?
    \item Als je aanneemt dat de oriëntatie van het object kan veranderen, wat is het verschil tussen maximum en minimum potentiële energie van het systeem?
\end{enumerate}

% \begin{description}[labelwidth=1.5cm, leftmargin=!]
%     \item[Geg. :]  
%     \item[Gevr. :]  
%     \item[Opl. :]   
% \end{description}

\begin{enumerate}[(a)]
    \item 
        \begin{description}[labelwidth=1.5cm, leftmargin=!]
            \item[Geg. :]  $q = -3.50$nC, $\Vec{r}_{+} = (-1.20\text{mm}, 1.10\text{mm})$, $\Vec{r}_{-} = (1.40\text{mm}, -1.30\text{mm})$
            \item[Gevr. :] $\Vec{p}$ ?
            \item[Opl. :]  $\Vec{p} = q\Vec{\ell} = q(\Vec{r}_{-} - \Vec{r}_{+}) 
                                  % = q((1.40\hat{i} - 1.30\hat{j}) - (-1.20\hat{i} + 1.10\hat{j})) 
                                    = q(2.60\hat{i} + - 2.40\hat{j}) \approx (-9.10\hat{i} + 8.40\hat{j}) \cdot 10^{-12}$ Cm
        \end{description}
        
    \item 
        \begin{description}[labelwidth=1.5cm, leftmargin=!]
            \item[Geg. :]  $\Vec{p} = (-9.10\hat{i} + 8.40\hat{j}) \cdot 10^{-12}$ Cm, $\Vec{E} = (7800\hat{i} - 4900\hat{j})$ N/C
            \item[Gevr. :] $\Vec{\tau}$ ?
            \item[Opl. :]  $\Vec{\tau} = \Vec{p} \times \Vec{E} = 
                            \begin{vmatrix}
                                \hat{i} & \hat{j} & \hat{k}\\ 
                                p_x & p_y & p_z\\
                                E_x & E_y & E_z  
                            \end{vmatrix} 
                            = 
                            \begin{vmatrix}
                                \hat{i} & \hat{j} & \hat{k}\\ 
                                -9.10 \cdot 10^{-12} & 8.40 \cdot 10^{-12} & 0\\
                                7800 & -4900 & 0  
                            \end{vmatrix} 
                            % =
                            % \begin{vmatrix}
                            %     -9.10 \cdot 10^{-12} & 8.40 \cdot 10^{-12}\\
                            %     7800 & -4900
                            % \end{vmatrix}\hat{k}
                            \approx -2.09 \cdot 10^{-8} \text{ Nm } \hat{k}$
        \end{description}
        
    \item 
        \begin{description}[labelwidth=1.5cm, leftmargin=!]
            \item[Geg. :]  $\Vec{p} = (-9.10\hat{i} + 8.40\hat{j}) \cdot 10^{-12}$ Cm, $\Vec{E} = (7800\hat{i} - 4900\hat{j})$ N/C
            \item[Gevr. :]  $U$?
            \item[Opl. :]   $U = -\Vec{p} \cdot \Vec{E} = -\sum_{i=1}^n p_iE_i \approx 1.12 \cdot 10^{-7}$Nm 
        \end{description}

    \item 
        \begin{description}[labelwidth=1.5cm, leftmargin=!]
            \item[Geg. :]  $\Vec{p} = (-9.10\hat{i} + 8.40\hat{j}) \cdot 10^{-12}$ Cm, $\Vec{E} = (7800\hat{i} - 4900\hat{j})$ N/C
            \item[Gevr. :]  $\Delta U$ ?
            \item[Opl. :]   $U_{\text{max}} = -pE\cos(0^{\circ}) = - pE$ \\
                            \hspace{0.5cm}$U_{\text{min}} = -pE\cos(180^{\circ}) = pE$ \\ 
                            \hspace{0.5cm}$\Delta U = U_{\text{max}} - U_{\text{min}} = -2pE \approx 2.28 \cdot 10^{-7}$Nm
        \end{description}

\end{enumerate}
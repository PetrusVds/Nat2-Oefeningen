\textbf{\underline{OZ 4 - De wet van Ampère en de wet van Biot-Savart - Oefening 1:}}
\vspace{0.5cm}

Een geleider bestaat uit een cirkelvormige lus met straal $R$ en twee lange rechte stukken. De draad ligt in het vlak van het blad en er loopt een stroom $I$ doorheen. Bepaal de grootte en de richting van het magnetische veld dat geproduceerd wordt in het centrum van de lus.

\begin{center}
    \includegraphics[scale = 0.5]{oz04/resources/Oz4Oef1.png}
\end{center}

\begin{description}[labelwidth=1.5cm, leftmargin=!]
    \item[Geg. :]   $I = 7.00$A, $R$
    \item[Gevr. :]  $\Vec{B}$ ?
    \item[Opl. :]  
    Stel nu dat $d\ell$ een infinitesimaal deeltje cirkelboog, dan is het infinitesimaal magnetisch veld door de lus
    \begin{align*}
        dB_L
            &= \frac{\mu_0I}{4\pi}\frac{d\ell}{R^2} \\
            &= \frac{\mu_0I}{4\pi}\frac{\sin(\gamma)}{R}d\theta \\
            &\overset{\perp}{=} \frac{\mu_0I}{4\pi}\frac{d\theta}{R} 
    \end{align*}
    waarbij $\gamma$ de hoek is tussen $\Vec{r}$ en $d\Vec{\ell}$ hierover integreren om het totale magnetische veld te bekomen
    \begin{align*}
        B_L
            &= \int_{0}^{2\pi} \frac{\mu_0I}{4\pi}\frac{d\theta}{R}  \\
            &= \frac{\mu_0I}{4\pi R} \int_{0}^{2\pi}d\theta \\
            &=  \frac{\mu_0I}{2R}
    \end{align*}
    Het bovenste punt zou twee keer moeten meegtelt worden, omdat er een overlap is (de andere overlap is niet loodrecht boven het punt en zal dus volgens de wet van ampere ons magnetisch veld niet beinvloeden). Dus tellen we er nog een factor
    \begin{equation*}
        B_P = \frac{\mu_0}{2\pi}\frac{I}{R}
    \end{equation*}
    bij en vervolgens krijgen we
    \begin{equation*}
        \Vec{B} = \Vec{B}_L + \Vec{B}_P = \frac{\mu_0I}{2R}\left(\frac{1}{\pi} + 1\right)(-\hat{k})
    \end{equation*}
    waarbij de rechterhandregel geeft dat het in het blad is.
\end{description}

\vspace{1cm}
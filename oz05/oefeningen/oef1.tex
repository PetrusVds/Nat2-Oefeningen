\textbf{\underline{OZ 5 - Magnetische inductie en de wet van Faraday - Oefening 1:}}
\vspace{0.5cm}

\begin{minipage}{0.76\textwidth}
    Een segment van een draad met lengte $d$ draagt een stroom $I$, zoals aangegeven in de figuur hiernaast.

    \begin{enumerate}[(a)]
        \item 
            Toon aan dat voor punten op de positieve $x$-as, zoals het punt $Q$, het magnetisch veld $\Vec{B}$ nul is.
        \item 
            Bepaal de uitdrukking van het magnetisch veld $\Vec{B}$ voor punten op de positieve $y$-as, zoals het punt $P$.
    \end{enumerate}
\end{minipage}
\begin{minipage}{0.2\textwidth}
    \begin{center}
        \includegraphics[scale = 0.35]{oz05/resources/Oz5Oef1.png}
    \end{center}
\end{minipage}

\begin{enumerate}[(a)]
    \item 
        \begin{description}[labelwidth=1.5cm, leftmargin=!]
            \item[Geg. :]  $Q$, $d$, $I$
            \item[Gevr. :] $\Vec{B}$
            \item[Opl. :]  
                We gebruiken de wet van Biot-Savart
                \begin{equation*}
                    \Vec{B} = \frac{\mu_0}{4\pi} \int \frac{Id\Vec{\ell} \times \hat{r}}{r^2} = \Vec{0}
                \end{equation*}
                waarbij $r$ en elke infinitesmiale $d\ell$ parallel zijn.
        \end{description}
    \item 
        \begin{description}[labelwidth=1.5cm, leftmargin=!]
            \item[Geg. :]  $P$, $d$, $I$
            \item[Gevr. :] $\Vec{B}$
            \item[Opl. :]   
                We gebruiken de wet van Biot-Savart
                \begin{align*}
                    \Vec{B} 
                        &= \frac{\mu_0}{4\pi} \int \frac{Id\Vec{\ell} \times \hat{r}}{r^2} \\
                        &= \frac{\mu_0I}{4\pi} \int \frac{d\Vec{\ell} \times \Vec{r}}{r^3} \\
                        &= \frac{\mu_0I}{4\pi} \int \frac{dx\hat{i}\times(-x\hat{i} + y\hat{j})}{(x^2 + y^2)^{\frac{3}{2}}} \\
                        &= \frac{\mu_0Iy}{4\pi} \int_0^d \frac{dx}{(x^2 + y^2)^{\frac{3}{2}}} \ (\hat{k}) \\
                        &= \frac{\mu_0Iy}{4\pi} \left[ \frac{x}{y^2(x^2 + y^2)^{\frac{1}{2}}} \right]_0^d \ (\hat{k}) \\
                        &= \frac{\mu_0I}{4\pi y} \frac{d}{(d^2 + y^2)^{\frac{1}{2}}} \ (\hat{k}) \\
                \end{align*}
        \end{description}
\end{enumerate}
    
\vspace{1cm}
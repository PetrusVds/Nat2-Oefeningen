\textbf{\underline{OZ 5 - Magnetische inductie en de wet van Faraday - Oefening 4:}}
\vspace{0.5cm}

Een toroïde met een gemiddelde straal van $20,0$ cm en $630$ windingen wordt opgevuld met staalpoeder dat een magnetische susceptibiliteit $\chi$ van $100$ heeft. Er wordt een stroom van $3,00$ A aangelegd. Bepaal het magnetisch veld dat geproduceerd wordt in de toroïde. (Je mag aannemen dat het magnetisch veld uniform is.)

\begin{description}[labelwidth=1.5cm, leftmargin=!]
    \item[Geg. :]  $r = 20.0$ cm, $N = 630$, $\chi = 100$, $I = 3.00$ A
    \item[Gevr. :] $\Vec{B}$
    \item[Opl. :]  
        We weten dat de totale magnetisch veld gegeven wordt door het volgende
        \begin{align*}
            B
                &= \mu_0 (1+\chi)H \\
                &= \mu_0 (1+\chi)(nI) \\
                &= \mu_0 (1+\chi)\left(\frac{N}{\ell}I\right) \\
                &= \mu_0 (1+\chi)\left(\frac{N}{2\pi r}I\right) \\
                &= 0.191 \text{ T}
        \end{align*}
\end{description}

\vspace{1cm}
\textbf{\underline{OZ 5 - Magnetische inductie en de wet van Faraday - Oefening 3:}}
\vspace{0.5cm}

Stel een metalen ring die vrij kan expanderen en samentrekken. De ring wordt in een constant magneetveld $\Vec{B}_0$ gebracht. Het magneetveld staat loodrecht op het vlak van de ring. De ring zal expanderen met een straal die lineair in de tijd toeneemt:
\begin{equation*}
    r(t) = r_0 + \alpha t.
\end{equation*}
De weerstand zal per lengte-eenheid van de ring toenemen volgens de empirische vergelijking:
\begin{equation*}
    R(\ell)  = R_{0}\ell(1 + \beta t)
\end{equation*}
Bereken de geïnduceerde stroom in de ring in functie van de tijd. Specificeer zowel de zin als de grootte van de stroom.

\begin{description}[labelwidth=1.5cm, leftmargin=!]
    \item[Geg. :]  $r(t)$, $R(\ell)$, $\Vec{B}_0$
    \item[Gevr. :]  $I_{\text{ind}}$
    \item[Opl. :] 
        \textbf{Opmerking:} Er is niet gegeven hoe het magneetveld gericht is. We nemen aan dat het magneetveld in het blad gaat, m\@.a\@.w\@.:
        \begin{equation*}
            \Vec{B}_0 = B_0 \ (-\hat{k}).
        \end{equation*}

        \noindent We gebruiken de wet van Faraday waarbij $d\vec{\ell}$ dezelfde richting heeft als $\vec{E}_{\text{ind}}$:
        \begin{equation*}
            \oint \Vec{E}_{\text{ind}} \cdot d\vec{\ell} = E_{\text{ind}}\int_0^{2\pi r(t)}d\ell= - \frac{d\Phi_B}{dt} 
        \end{equation*}
        waarbij
        \begin{equation*}
            E_{\text{ind}} 
                = \rho J 
                % = \left(\frac{A}{\ell}R(\ell)\right)\frac{I_{\text{ind}}}{A} 
                = I_{\text{ind}}\left(R_{0}(1 + \beta t)\right)
        \end{equation*} 
        We krijgen nu
        \begin{equation*}
                I_{\text{ind}}\left(R_{0}(1 + \beta t)\right) \int_0^{2\pi r(t)} d\ell = - \frac{d\Phi_B}{dt}
        \end{equation*}
        waaruit volgt, waarbij $\vec{A} = A \ (\hat{k})$:
        \begin{align*}
            I_{\text{ind}} 
                &= \frac{\frac{d}{dt} \left( B_0 A \right)}{\left(R_{0}(1 + \beta t)\right) \int_0^{2\pi r(t)} d\ell } \\
                &= \frac{\frac{d}{dt} \left( B_0 A \right)}{\left(R_{0}(1 + \beta t)\right) \left(2\pi r(t)\right) } \\
                % &= \frac{\frac{d}{dt} \left(B_0 \pi r(t)^2 \right)}{\left(R_{0}(1 + \beta t)\right) \left(2\pi r(t)\right)} \\
                &= \frac{B_0 \left(\frac{d}{dt} \pi r(t)^2\right)}{\left(R_{0}(1 + \beta t)\right)\left(2\pi r(t)\right)} \\
                &= \frac{B_0 \left(2\pi r(t)\right) \left(\frac{d}{dt} \alpha t\right)}{\left(R_{0}(1 + \beta t)\right)\left(2\pi r(t)\right)} \\
                &= \frac{B_0 \alpha}{\left(R_{0}(1 + \beta t)\right)}
        \end{align*}
        De stroom zal wijzerzin gaan als $\alpha > 0$.
\end{description}

\vspace{1cm}
\textbf{\underline{OZ 5 - Magnetische inductie en de wet van Faraday - Oefening 5:}}
\vspace{0.5cm}

Een geleidende staaf (massa $m$, weerstand $R$) rust op twee wrijvingsloze en weerstandsloze parallelle rails (afstand $\ell$ tussen de twee rails) in een uniform magnetisch veld $\Vec{B}$, zie figuur hieronder. Op tijdstip $t = 0$ is de staaf in rust en is er een spanningsbron verbonden aan de punten $a$ en $b$. 

\begin{enumerate}[(a)]
    \item 
        Bepaal de snelheid van de staaf in functie van de tijd als een constante stroombron wordt gebruikt.
    \item 
        Bepaal de snelheid van de staaf in functie van de tijd als constante spanningsbron (emf) gebruikt wordt.
    \item 
        Bereikt de staaf een eindige snelheid? Indien ja, wat is deze snelheid dan?
\end{enumerate}

\begin{center}
    \includegraphics[scale = 0.35]{oz05/resources/Oz5Oef5.png}
\end{center}

\begin{enumerate}[(a)]
    \item 
        \begin{description}[labelwidth=1.5cm, leftmargin=!]
            \item[Geg. :]  $m$, $R$, $\ell$ $B$
            \item[Gevr. :] $v(t)$
            \item[Opl. :]  
                We weten de formule voor magnetische kracht
                \begin{equation*}
                    F = I\ell B = m a = m \frac{dv}{dt} 
                \end{equation*}
                waaruit volgt 
                \begin{equation*}
                    v(t) = \frac{I \ell B}{m} t
                \end{equation*}
        \end{description}

    \newpage
    
    \item 
        \begin{description}[labelwidth=1.5cm, leftmargin=!]
            \item[Geg. :]  $m$, $R$, $\ell$ $B$
            \item[Gevr. :] $v(t)$
            \item[Opl. :]  
                De magnetische flux is
                \begin{equation*}
                    \Phi_B = \int \Vec{B} \cdot d\Vec{A} = B \ell x
                \end{equation*}
                waaruit we de geïnduceerde emf bepalen
                \begin{equation*}
                    \mathcal{E}_{\text{ind}} = - \frac{d\Phi_B}{dt} = - B\ell \frac{dx}{dt} = - B\ell v
                \end{equation*}
                wat betekent dat de geïnduceerde stroom het volgende is
                \begin{equation*}
                    I_{\text{ind}} = \frac{Blv}{R}.
                \end{equation*}
                We weten de formule voor magnetische kracht
                \begin{align*}
                    F_{I} 
                        &=  I \ell B \\
                    F_{I_{\text{ind}}} 
                        &= I_{\text{ind}}\ell B = \frac{B^2 \ell^2 v}{R} 
                \end{align*}
                waarop we de tweede wet van Newton toepassen
                \begin{equation*}
                    F_{\text{net}} = F_{I} - F_{I_{\text{ind}}} = I\ell B - \frac{B^2 \ell^2 v}{R} = m\frac{dv}{dt}
                \end{equation*}
                wat leidt tot de differentiaalvergelijking
                \begin{equation*}
                      \frac{dv}{dt} - \frac{B^2 \ell^2}{mR} v + \frac{I \ell B}{m} = 0
                \end{equation*}
                wat als oplossing heeft
                \begin{equation*}
                    v(t) = \frac{\epsilon_0}{B\ell}\left(1-e^{\left(\frac{-B^2\ell^2t}{mR}\right)}\right)
                \end{equation*}
        \end{description}
    \item 
        \begin{description}[labelwidth=1.5cm, leftmargin=!]
            \item[Geg. :]  $m$, $R$, $\ell$ $B$
            \item[Gevr. :] $\lim_{t\to\infty} v(t)$
            \item[Opl. :]  
                We berekenen de limiet tot oneindig van de gevonden $v(t)$ uit (b), we krijgen
                \begin{align*}
                    \lim_{t\to\infty} v(t) 
                        &= \lim_{t\to\infty} \frac{\epsilon_0}{B\ell}\left(1-e^{\left(\frac{-B^2\ell^2t}{mR}\right)}\right) \\
                        &= \frac{\epsilon_0}{B\ell}\left(1- \lim_{t\to\infty} e^{\left(\frac{-B^2\ell^2t}{mR}\right)}\right) \\
                        &= \frac{\epsilon_0}{B\ell}\left(1- 0\right) \\
                        &= \frac{\epsilon_0}{B\ell}
                \end{align*}
        \end{description}
\end{enumerate}


\vspace{1cm}
% \textbf{\underline{OZ 8 - $ LC $- en $ RLC $-circuits - Oefening 5:}}
% \vspace{0.5cm}

% Een condensator, een spoel, een weerstand en een wisselstroombron worden op verschillende manieren aan elkaar geschakeld. Eerst wordt de condensator met de stroombron verbonden, er ontstaat dan een rms stroom van 25,1 mA. Vervolgens wordt de condensator uit het circuit gehaald, ontladen en opnieuw aan de stroombron gehangen in serie met de weerstand. De rms stroom die dan gaat vloeien is 15,7 mA. Daarna wordt de schakeling afgebroken, de condensator wordt opnieuw ontladen en dan schakelt met de condensator, de spoel en de stroombron in serie. De rms stroom die dan gaat lopen is 68,2 mA. Ten slotte ontlaadt men de condensator opnieuw en schakelt men alle componenten in serie. Wat is de rms stroom die nu gaat lopen?

% \begin{description}[labelwidth=1.5cm, leftmargin=!]
%     \item[Geg. :]   $ I_{C} = 25,1 $ mA; $ I_{RC} = 15,7 $ mA; $ I_{LC} = 68,2 $ mA;
%     \item[Gevr. :]  $ I_{RLC} $;
%     \item[Opl. :] (Oplossing van Serway 6E Chapter 33 Problem 24)

%     For the source-capacitor circuit, the rms source voltage is $\Delta V_s=(25.1 \mathrm{~mA}) X_C$. For the circuit with resistor, $\Delta V_s=(15.7 \mathrm{~mA}) \sqrt{R^2+X_C^2}=(25.1 \mathrm{~mA}) X_C$. This gives $R=1.247 X_C$. For the circuit with ideal inductor, $\Delta V_s=(68.2 \mathrm{~mA})\left|X_L-X_C\right|=(25.1 \mathrm{~mA}) X_C$. So $\left|X_L-X_C\right|=0.3680 X_C$. Now for the full circuit
% $$
% \begin{aligned}
% & \Delta V_s=I \sqrt{R^2+\left(X_L-X_C\right)^2} \\
% & (25.1 \mathrm{~mA}) X_C=I \sqrt{\left(1.247 X_C\right)^2+\left(0.368 X_C\right)^2} \\
% & I=19.3 \mathrm{~mA}
% \end{aligned}
% $$
% \end{description}
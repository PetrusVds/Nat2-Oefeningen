% \textbf{\underline{OZ 8 - $ LC $- en $ RLC $-circuits - Oefening 1:}}
% \vspace{0.5cm}

% Een koperen draad van 200 m lang en 1,00 mm diameter wordt rond een plastieken buis gewonden om een laange solenoïde te maken. De spoel heeft een cirkelvormige doorsnede en bestaat uit 1 laag van dicht opeengepakte windingen. Als de stroom in deze spoel lineair afneemt van 1,80 A naar 0 A in 0,120 s, wordt er een spanning geïnduceerd van 80,0 mV. Wat is de lengte van de solenoïde (gemeten langs de as)?

% \begin{description}[labelwidth=1.5cm, leftmargin=!]
%     \item[Geg. :]   $ l_{draad} = 200 $ m; $ d_{draad} = 1,00 $ mm; $ I_0 = 1,80 $ A; $ I_e = 0 $ A; $ t_e = 0,120 $ s; $ \varepsilon = 80,0 $ mV;
%     \item[Gevr. :]  $ l_{solenoide} $;
%     \item[Opl. :]   $ N = \dfrac{l_{draad}}{2 \pi R} $
    
%                     $ B = \dfrac{\mu_0 N I}{l_{solenoide}} $
                    
%                     $ \Phi_B = \int{\vec{B} \cdot d\vec{A}} 
%                     = B \cdot A 
%                     = \dfrac{\mu_0 N I}{l_{solenoide}} \cdot \pi \cdot R^2 $
                    
%                     $ L = \dfrac{N \Phi_B}{I} = \dfrac{\mu_0 N^2 \pi R^2}{l_{solenoide}} 
%                     = \dfrac{\mu_0 {\left( \dfrac{l_{draad}}{2 \pi R} \right)}^2 \pi R^2}{l_{solenoide}} 
%                     = \dfrac{\mu_0 l_{draad}^2}{4 \pi l_{solenoide}} $
                    
%                     $ \varepsilon = L \dfrac{\Delta I}{\Delta t} 
%                     = \dfrac{\mu_0 l_{draad}^2}{4 \pi l_{solenoide}} \dfrac{\Delta I}{\Delta t} $
                    
%                     \hspace{-0.57cm} $ \Leftrightarrow 
%                     l_{solenoide} = \dfrac{\mu_0 l_{draad}^2}{4 \pi \varepsilon} \dfrac{\Delta I}{\Delta t} 
%                     = \dfrac{4 \pi \cdot 10^{-7} \cdot 200^2}{4 \pi \cdot 80,0 \cdot 10^{-3}} \dfrac{1,80 - 0}{0,120 - 0} 
%                     = 0,750 $ m
% \end{description}

% \vspace{1cm}
% \textbf{\underline{OZ 8 - $ LC $- en $ RLC $-circuits - Oefening 4:}}
% \vspace{0.5cm}

% De energie van een $ RLC $-circuit (zonder stroombron) vermindert met 1,00\% gedurende elke oscillatie als $ R = 2,00 \ \Omega $. Als deze weerstand wordt verwijderd, oscilleert het resulterende $ LC $-circuit met 1,00 kHz. Bepaal de waarden van de inductantie en de capaciteit in het $ RLC $-circuit.

% \begin{description}[labelwidth=1.5cm, leftmargin=!]
%     \item[Geg. :]   $ R = 2,00 \ \Omega $; $ f = 1,00 kHz $; $ U = U_0 \cdot 0,99^{t/T} $;
%     \item[Gevr. :]  $ L $; $ C $;
%     \item[Opl. :] (Oplossing van Serway 6E Chapter 32 Problem 57)
    
%     The period of damped oscillation is $T=\frac{2 \pi}{\omega_d}$. After one oscillation the charge returning to the capacitor is $Q=Q_{\max } e^{-R T / 2 L}=Q_{\max } e^{-2 \pi R / 2 L \omega_d}$. The energy is proportional to the charge squared, so after one oscillation it is $U=U_0 e^{-2 \pi R / L \omega_d}=0.99 U_0$. Then
% $$
% \begin{aligned}
% & e^{2 \pi R / L \omega_d}=\frac{1}{0.99} \\
% & \frac{2 \pi 2 \Omega}{L \omega_d}=\ln (1.0101)=0.01005 \\
% & L \omega_d=\frac{2 \pi 2 \Omega}{0.01005}=1250 \Omega=L\left(\frac{1}{L C}-\frac{R^2}{4 L^2}\right)^{1 / 2} \\
% & 1.563 \times 10^6 \Omega^2=\frac{L}{C}-\frac{(2 \Omega)^2}{4} \\
% & \frac{L}{C}=1.563 \times 10^6 \Omega^2
% \end{aligned}
% $$
% We are also given
% $$
% \begin{aligned}
% & \omega=2 \pi \times 10^3 / \mathrm{s}=\frac{1}{\sqrt{L C}} \\
% & L C=\frac{1}{\left(2 \pi \times 10^3 / \mathrm{s}\right)^2}=2.533 \times 10^{-8} \mathrm{~s}^2
% \end{aligned}
% $$
% Solving simultaneously,
% $$
% \begin{aligned}
% & C=2.533 \times 10^{-8} \mathrm{~s}^2 / L \\
% & \frac{L^2}{2.533 \times 10^{-8} \mathrm{~s}^2}=1.563 \times 10^6 \Omega^2 \quad L=0.199 \mathrm{H} \\
% & C=\frac{2.533 \times 10^{-8} \mathrm{~s}^2}{0.199 \mathrm{H}}=127 \mathrm{nF}=\mathrm{C}
% \end{aligned}
% $$

% \end{description}

% \vspace{1cm}
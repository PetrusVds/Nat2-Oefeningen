\textbf{\underline{OZ 8 - LC- en  RLC-circuits - Oefening 4:}}
\vspace{0.5cm}

Beschouw de kring met een spanningsbron met emf van $50.0 \ \text{V}$, een weerstand van $250 \ \Omega$, en een capaciteit van $0.500 \ \mu \text{F}$. De schakelaar S is gesloten voor een lange tijd en geen spanningsverschil wordt gemeten over de condensator. Nadat de schakelaar geopend wordt, bereikt het potentiaal verschil over de condensator een maximale waarde van $150.0 \ \text{V}$. Wat is dan de inductantie $L$ in de kring?

\begin{center}
    \includegraphics[scale = 0.3]{oz08/resources/Oz8Oef4.png}
\end{center}
    
\begin{description}[labelwidth=1.5cm, leftmargin=!]
    \item[Geg. :]  $\mathcal{E} = 50.0 \ \text{V}$, $R = 250 \ \Omega$, $C = 0.500 \ \mu \text{F}$, $\Delta V_{\text{max}} = 150.0 \ \text{V}$
    \item[Gevr. :] $L$ ?
    \item[Opl. :] 
        Stel de schakelaar is lang gesloten geweest, dan is de stroom 
        \begin{equation*}
            I_{\text{max}} = \frac{\mathcal{E}}{R}
        \end{equation*}
        want de condensator is dan volledig opgeladen. Sinds de potentiaal verschillen over de inductor en de condensator gelijk zijn, dan ook de potentiële energie opgeslagen in de inductor en de condensator:
        \begin{equation*}
            \frac{1}{2}LI_{\text{max}}^2 = \frac{1}{2}C\Delta V_{\text{max}}^2. 
        \end{equation*}
        We kunnen nu de inductantie berekenen:
        \begin{equation*}
            L = \frac{C\Delta V_{\text{max}}^2}{I_{\text{max}}^2} = \frac{C\Delta V_{\text{max}}^2R^2}{\mathcal{E}^2} \approx 281 \ \text{mH}.
        \end{equation*}

\end{description}

\vspace{1cm}
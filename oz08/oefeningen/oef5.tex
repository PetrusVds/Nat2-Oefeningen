\textbf{\underline{OZ 8 - LC- en  RLC-circuits - Oefening 5:}}
\vspace{0.5cm}

Beschouw een LC-kring waarbij $L = 500 \ \text{mH}$ en $C = 0.100 \ \mu \text{F}$. 

\begin{enumerate}[(a)]
    \item 
        Wat is de resonantie frequentie $\omega_0$?
    \item 
        Indien een weerstand R van $1.00 \ \text{k}\Omega$ wordt toegevoegd aan het circuit, wat is dan de frequentie van de (gedempte) oscillaties?
\end{enumerate}

\begin{description}[labelwidth=1.5cm, leftmargin=!]
    \item[Geg. :]   $L = 500 \ \text{mH}$, $C = 0.100 \ \mu \text{F}$
\end{description}

\begin{enumerate}[(a)]
    \item 
        \begin{description}[labelwidth=1.5cm, leftmargin=!]
            \item[Gevr. :] $\omega_0$ ?
            \item[Opl. :]   
                De resonantie frequentie in een LC-kring wordt gegeven door de volgende formule:
                \begin{equation*}
                    \omega_0 = \frac{1}{\sqrt{LC}} \approx 4.47 \cdot 10^3 \ \text{rad/s}.
                \end{equation*}
        \end{description}
    \item 
        \begin{description}[labelwidth=1.5cm, leftmargin=!]
            \item[Geg. :]   $R = 1.00 \ \text{k}\Omega$
            \item[Gevr. :] $\omega$ ?
            \item[Opl. :]   
                De frequentie van de gedempte oscillaties wordt gegeven door de volgende formule:
                \begin{equation*}
                    \omega = \sqrt{\omega_0^2 - \frac{R^2}{4L^2}} \approx 4.36 \cdot 10^3 \ \text{rad/s}.
                \end{equation*}
        \end{description}
\end{enumerate}

\vspace{1cm}
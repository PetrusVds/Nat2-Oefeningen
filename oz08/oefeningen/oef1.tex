\textbf{\underline{OZ 8 - LC- en  RLC-circuits - Oefening 1:}}
\vspace{0.5cm}

Op $t = 0$ s wordt een emf van $500$ V aangelegd op een spoel met een inductantie van
$0.800$ H en een weerstand $30 \ \Omega$.

\begin{enumerate}[(a)]
    \item Bepaal de energie opgeslagen in het magnetisch
    veld wanneer de stroom de helft van zijn maximale waarde heeft bereikt
    \item Nadat de emf wordt aangesloten, hoe lang duurt het voordat de helft van de maximale
    stroom bereikt wordt?
\end{enumerate}

\begin{description}[labelwidth=1.5cm, leftmargin=!]
    \item[Geg. :]  $L = 0.800 \ \text{H}$, $R = 30 \ \Omega$, $V = 500 \ \text{V}$
\end{description}

\begin{enumerate}[(a)]
    \item 
        \begin{description}[labelwidth=1.5cm, leftmargin=!]
            \item[Geg. :] $I = \frac{1}{2}I_{\text{max}}$
            \item[Gevr. :] $U_L$ ?
            \item[Opl. :]  
                De maximale stroom is triviaal te berekenen:
                \begin{equation*}
                    I_{\text{max}} = \frac{V}{R}
                \end{equation*}
                Als we nu de energie willen berekenen op het moment dat de stroom de helft van zijn maximale waarde heeft bereikt, dan gebruiken we de volgende formule:
                \begin{equation*}
                    U_L = \frac{1}{2} L I^2 = \frac{1}{2}L\left(\frac{1}{2} I_{\text{max}}\right)^2 =  \frac{1}{8}\frac{LV^2}{R^2} = 27.8 \ \text{J}.
                \end{equation*}
        \end{description}
    \item 
        \begin{description}[labelwidth=1.5cm, leftmargin=!]
            \item[Gevr. :] $t_{50\%}$ ?
            \item[Opl. :]   
                De stroom in een RL-kring is gegeven door:
                \begin{equation*}
                    I(t) = I_{\text{max}}\left(1 - e^{-\frac{R}{L}t}\right)
                \end{equation*}
                We zoeken nu de tijd $t_{50\%}$ (na een paar triviale herwerkingsstappen) waarvoor geldt dat 
                \begin{equation*}
                    e^{-\frac{R}{L}t_{50\%}} = \frac{1}{2}
                \end{equation*}
                wat we makkelijk oplossen door het natuurlijke logaritme te nemen van beide kanten:
                \begin{equation*}
                    t_{50\%} = -\frac{L}{R}\ln\left(\frac{1}{2}\right) = 18.5 \ \text{ms}.
                \end{equation*}
        \end{description}
\end{enumerate}



\vspace{1cm}
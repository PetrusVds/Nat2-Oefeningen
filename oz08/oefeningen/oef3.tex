% \textbf{\underline{OZ 8 - $ LC $- en $ RLC $-circuits - Oefening 3:}}
% \vspace{0.5cm}

% Beschouw een $ LC $-circuit bestaande uit een parallelle-platen condensator met oppervlakte $ A $ en een spoel met inductantie $ L $. Dit circuit kan op de volgende manier gebruikt worden om kleine afstanden te bepalen:

% \begin{enumerate}[(a)]
%     \item Indien een lading in dit circuit oscilleert met frequentie $ f $ als de platen van de condensator zich op een afstand $ x $ van elkaar bevinden, toon dan aan dat $ x = 4 \pi^2 A \varepsilon_0 f^2 L $.
%     \item Wanneer we de afstand aanpassen met $ \Delta x $, verandert de frequentie met $ \Delta f $. Toon aan dat $ \Delta x/x \approx 2 \left( \Delta f/f \right) $.
% \end{enumerate}

% \begin{description}[labelwidth=1.5cm, leftmargin=!]
%     \item[Geg. :]   $ A $; $ L $;
% \end{description}

% \begin{enumerate}[(a)]
%     \item 
%         \begin{description}[labelwidth=1.5cm, leftmargin=!]
%             \item[Geg. :]   $ f $; $ x $;
%             \item[Gevr. :]  Toon aan dat $ x = 4 \pi^2 A \varepsilon_0 f^2 L $;
%             \item[Opl. :]   $ C = \dfrac{\varepsilon_0 A}{x} $
            
%                             $ I X_C = I X_L $
                            
%                             \hspace{-0.57cm} $ \Leftrightarrow 
%                             \dfrac{1}{\omega C} = \omega L $
                            
%                             \hspace{-0.57cm} $ \Leftrightarrow 
%                             C = \dfrac{1}{\omega^2 L} $
                            
%                             \hspace{-0.57cm} $ \Leftrightarrow 
%                             \dfrac{\varepsilon_0 A}{x} = \dfrac{1}{\omega^2 L} $
                            
%                             \hspace{-0.57cm} $ \Leftrightarrow
%                             x = \varepsilon_0 A \omega^2 L 
%                             = \varepsilon_0 A (2 \pi f)^2 L 
%                             = 4 \pi^2 A \varepsilon_0 f^2 L $
%         \end{description}
%     \item
%         \begin{description}[labelwidth=1.5cm, leftmargin=!]
%             \item[Geg. :]   $ x $; $ \Delta x $; $ f $; $ \Delta f$;
%             \item[Gevr. :]  Toon aan dat $ \Delta x/x \approx 2 \left( \Delta f/f \right) $;
%             \item[Opl. :]   $ \dfrac{\Delta x}{x} = \dfrac{x_e - x}{x} 
%                             = \dfrac{4 \pi^2 A \varepsilon_0 f_e^2 L - 4 \pi^2 A \varepsilon_0 f^2 L}{4 \pi^2 A \varepsilon_0 f^2 L}
%                             = \dfrac{f_e^2 - f^2}{f^2}
%                             = \dfrac{\left( f_e - f \right) \left( f_e + f \right)}{f^2} $
                            
%                             \vspace{0.5cm}
                            
%                             \begin{quote}
%                                 Kleine verandering $ \to f_e + f \approx 2f $
%                             \end{quote}
                            
%                             \vspace{0.5cm}
                            
%                             $ \dfrac{\Delta x}{x} \approx \dfrac{\left( f_e - f \right) 2f}{f^2} = 2 \dfrac{f_e - f}{f} $
                            
%                             \hspace{-0.57cm} $ \Rightarrow
%                             \dfrac{\Delta x}{x} \approx 2 \dfrac{\Delta f}{f} $
%         \end{description}
% \end{enumerate}

% \vspace{1cm}
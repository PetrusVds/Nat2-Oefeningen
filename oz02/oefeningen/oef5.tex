\textbf{\underline{OZ 2 - Magnetische velden - Oefening 5:}}
\vspace{0.5cm}

Bereken het magnetisch dipoolmoment van een roterende geladen isolerende schijf
met straal $R$, waarbij de schijf rond zijn symmetrieas draait met een hoeksnelheid $\omega$
en de ladingsdichtheid varieert met de straal $r$ volgens $ \sigma = c \cdot r$, met $c$ een constante.

% \begin{description}[labelwidth=1.5cm, leftmargin=!]
%     \item[Geg. :]   
%     \item[Gevr. :]  
%     \item[Opl. :]  
% \end{description}


\begin{description}[labelwidth=1.5cm, leftmargin=!]
    \item[Geg. :]   $R,\omega, r, \sigma$
    \item[Gevr. :]  $\mu$
    \item[Opl. :]   
    
                    We stellen de formule van het infinitesimale oppervlakte op
                    \begin{align}
                         dA &= 2\pi rdr \nonumber \\ 
                        \intertext{waaruit we de formule voor de infinitesimale lading kunnen afleiden}
                         dq &= \sigma dA \nonumber \\ 
                            &= c2\pi r^2dr \\
                        \intertext{We weten de formule van de periode van de cirkelbeweging}
                          T &= \dfrac{2\pi}{\omega}  \\ 
                        \intertext{Uit (1) en (2) vinden we}
                         dI &= \dfrac{dq}{dt} \nonumber \\ 
                            &= \dfrac{dq}{T} \nonumber \\ 
                            &= \omega cr^2dr \nonumber \\
                        \intertext{waaruit we de infinitesimale magnetisch dipoolmoment kunnen halen}
                     d\mu &= dI(\pi r^2) \nonumber \\ 
                            % &= (\omega cr^2dr)(\pi r^2) \\ 
                            &= \pi\omega cr^4 dr \nonumber
                        \intertext{en dus het magnetisch dipoolmoment}
                        \mu &= \int_0^R d\mu \nonumber \\ 
                            &= \pi\omega \int_0^R r^4 dr \nonumber \\ 
                            &= \pi\omega c\tfrac{R^5}{5} \nonumber
                    \end{align}
\end{description} 

\vspace{1cm}


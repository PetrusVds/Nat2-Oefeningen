\textbf{\underline{OZ 2 - Magnetische velden - Oefening 3:}}
\vspace{0.5cm}

% \begin{description}[labelwidth=1.5cm, leftmargin=!]
%     \item[Geg. :]   
%     \item[Gevr. :]  
%     \item[Opl. :]  
% \end{description}
Een stroom $I$ vloeit door een rechthoekige balk van geleidend materiaal terwijl een magnetisch veld $\Vec{B}$ is aangelegd zoals in de figuur. 

\begin{enumerate}[(a)]
    \item Als de ladingsdragers \textbf{positief} zijn, in welke richting worden ze dan afgebogen door het magnetisch veld? Deze afbuiging zorgt ervoor dat de bovenkant en de onderkant van de balk een netto lading krijgen. Dit produceert een elektrisch veld dat het effect van de magnetische kracht gaat tegenwerken. Het systeem is in evenwicht als de twee krachten elkaar opheffen.
    \item Wat is het potentiaalverschil tussen de boven- en onderkant van de balk als het systeem in evenwicht is? Druk je resultaat uit in functie van $B$, $v$ (de snelheid van de ladingsdragers) en de relevante afmetingen van de balk. (Vermeld ook welke kant de hoogste potentiaal heeft.)
    \item Wat zou er veranderen aan je antwoorden als de ladingsdragers negatief zouden zijn maar de stroom nog steeds in dezelfde richting zou lopen?
\end{enumerate}

\begin{enumerate}[(a)]
    \item Het zou uit het blad bewegen.
    \item 
        \begin{description}[labelwidth=1.5cm, leftmargin=!]
        \item[Gevr. :]  $\mathcal{E}_{H}$
        \item[Opl. :]   $\mathcal{E}_{H} = E_Hd = vBd $
        \end{description}
    \item De Hallspanning zou gelijk zijn, maar de ladingen zouden in het blad bewegen.
\end{enumerate}

\vspace{1cm}
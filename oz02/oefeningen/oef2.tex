\textbf{\underline{OZ 2 - Magnetische velden - Oefening 2:}}
\vspace{0.5cm}

Een uniform magnetisch veld van $ 0,150 $ T ligt volgens de x-as. Een positron vliegt dit veld binnen met een snelheid van $5, 00 \cdot 10^6 $ m/s onder een hoek van $85,0^{\circ}$ met de x-as (de massa van een positron is $9, 11 \cdot 10^{-31} $ kg). Bereken de straal van de helix die
het positron gaat maken en ook de afstand p tussen twee opeenvolgende windingen van de helix.

% \begin{description}[labelwidth=1.5cm, leftmargin=!]
%     \item[Geg. :]   
%     \item[Gevr. :]  
%     \item[Opl. :]  
% \end{description}

\begin{description}[labelwidth=1.5cm, leftmargin=!]
    \item[Geg. :]   $ B = 0,150 $ T ,
                    $ v = 5, 00 \cdot 10^6 $ m/s,
                    $ \theta = 85,0^{\circ} $,
                    $ m = 9, 11 \cdot 10^{-31} $ kg                 
    \item[Gevr. :]  $ r, p$
    \item[Opl. :]   $ F_{B} = qvB\sin(\theta) = 1.20 \cdot 10^{-13} $ 
                    \vspace{0.3cm}\\ 
                    \hspace{-0.57cm} $ a_{R} = \dfrac{F_{B}}{m} = \dfrac{v^2}{r}$ \vspace{0.3cm}\\ 
                    \hspace{-0.57cm} $ r = \dfrac{mv^2}{F_{B}} = 1.89 \cdot 10^{-4}$
                    \vspace{0.3cm}\\ 
                    \hspace{-0.57cm} $ p = 2\pi r \cos(85^{\circ}) = 1.04 \cdot 10^{-4}$
                    
\end{description}

\vspace{1cm}
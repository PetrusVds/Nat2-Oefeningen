\textbf{\underline{OZ 2 - Magnetische velden - Oefening 1:}}
\vspace{0.5cm}

Beschouw een massa separator met $E = 2, 48 \cdot 10^{4} $V/m en $B_{\text{in}} = B_{0,\text{in}} = 0,680 $T. We sturen nu koolstofionen met massagetallen 12, 13 en 14 door deze massa separator. Hoe ver liggen de lijnen van de verschillende (enkel geladen) isotopen uit elkaar op de detector? Wat als de ionen 2+ geladen zijn?

\begin{description}[labelwidth=1.5cm, leftmargin=!]
    \item[Geg. :]   $E = 2, 48 \cdot 10^{4} $ V/m, $B_{\text{in}} =         B_{0,\text{in}} = 0,680 $T, $ m_{1} = 12(1.67 \cdot 10^{-27})$ kg,
                    $ m_{2} = 13(1.67 \cdot 10^{-27})$ kg,
                    $ m_{3} = 14(1.67 \cdot 10^{-27})$ kg,
                    $q = 1.60 \cdot 10^{-19}$C
    \item[Gevr. :]  $d_{1 \to 2}, d_{2 \to 3}, d_{3 \to 1}$
    \item[Opl. :]  
                    $r_{1} = \dfrac{m_{1}v_{1}}{qB_{0,\text{in}}} =  \dfrac{m_{1}E}{qB_{\text{in}}B_{0,\text{in}}} = 6.67 \cdot 10^{-3}$ \vspace{0.3cm}\\
                    \hspace{0.57cm} $r_{2} = \dfrac{m_{2}v_{2}}{qB_{0,\text{in}}} =  \dfrac{m_{2}E}{qB_{\text{in}}B_{0,\text{in}}} = 7.23 \cdot 10^{-3}$ 
                    \vspace{0.3cm}\\
                    \hspace{0.57cm} $r_{3} = \dfrac{m_{3}v_{3}}{qB_{0,\text{in}}} =  \dfrac{m_{2}E}{qB_{\text{in}}B_{0,\text{in}}} = 7.79 \cdot 10^{-3}$ 
                    \vspace{0.3cm}\\
                    \hspace{0.57cm} $d_{1 \to 2} = 2r_{2} - 2r_{1} = 1.10 \cdot 10^{-3} $ 
                    \vspace{0.3cm}\\
                    \hspace{0.57cm} $d_{2 \to 3} = 2r_{3} - 2r_{2} = 1.12 \cdot 10^{-3} $ 
                    \vspace{0.3cm}\\
                    \hspace{0.57cm} $d_{3 \to 1} = 2r_{3} - 2r_{1} = 2.22 \cdot 10^{-3} $ 
                    \vspace{0.3cm}\\
                    Als we nu de lading 2 keer vergroten, dan zien we dat de straal 2 keer verkleint en dus halveert het verschilt tussen de lijnen. 

\end{description}

\vspace{1cm}
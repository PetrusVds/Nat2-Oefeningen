\textbf{\underline{OZ 6 - Magnetische inductie en de wet van Faraday - Oefening 3:}}
\vspace{0.5cm}

    \begin{minipage}{.76\textwidth}
        Een homogeen magnetisch veld wordt aangelegd in de cirkel. Het veld verandert in de tijd volgens $B = (2.00t^3 - 4.00t^2 +0.800) $ T met $t$ de tijd in seconden. Zij $r_2 = 2R = 5.00$ cm.

        \begin{enumerate}[(a)]
            \item Bereken de grootte en de richting van de kracht die inwerkt op een elektron dat zich in een punt $P_2$ bevindt als $t = 2.00$ s.
            \item Op welk tijdstip is deze kracht gelijk aan $0$?
        \end{enumerate}    
    \end{minipage}
    \hspace{0.5cm}\begin{minipage}{.2\textwidth}
        \includegraphics[scale = 0.25]{oz06/resources/Oz6Oef3.png}
    \end{minipage}

    \begin{enumerate}[(a)]
        \item     
            \begin{description}[labelwidth=1.5cm, leftmargin=!]
                \item[Geg. :] $B = (2.00t^3 - 4.00t^2 +0.800) $ T, $r_2 = 2R = 5.00$ cm, $t = 2.00$ s, $q = -1.60 \cdot 10^{-19}$ C
                \item[Gevr. :] $F$
                \item[Opl. :] 
                    We weten volgens de wet van Faraday dat waarbij we $d\vec{s}$ kiezen in tegenwijzerzin (en dus $d\vec{A}$ uit het bord)
                    \begin{equation*}
                        \oint \vec{E} \cdot d\vec{s} =  - \frac{d}{dt} \left( \int \vec{B} \cdot d\vec{A} \right)
                    \end{equation*}
                    % en dus dat
                    % \begin{equation*}
                    %     \oint E dr = \frac{d}{dt} \left( \int B dA \right) = \pi r^2 \frac{dB}{dt} 
                    % \end{equation*}
                    ofwel dat 
                    \begin{equation*}
                        E(t) = \frac{\pi (\frac{r_2}{2})^2}{2\pi r_2}\frac{d\left((2.00t^3 - 4.00t^2 +0.800)\right)}{dt} = \frac{r_2}{8} (6.00t^2 - 8.00t).
                    \end{equation*}
                    Hieruit volgt voor $t = 2.00$ s dat
                    \begin{equation*}
                        F(q,t) = qE(t) = - 8.00 \cdot 10^{-21}.
                    \end{equation*}
                    Vectorieel wordt dit
                    \begin{equation*}
                        \vec{F}_{P_2} = 8.00 \cdot 10^{-21} \ (-\hat{j})
                    \end{equation*}
            \end{description}
        \item     
            \begin{description}[labelwidth=1.5cm, leftmargin=!]
                \item[Geg. :] $F(t) = 0$
                \item[Gevr. :] $t$
                \item[Opl. :]
                    We zoeken de nulpunten van de functie 
                    \begin{equation*}
                        F(q,t) = qE(t) = q \left( \frac{r_2}{8} (6.00t^2 - 8.00t) \right) = 0.
                    \end{equation*}
                    De enigste factor die nul kan worden is $6.00t^2 - 8.00t$ en dus zoeken we de nulpunten hiervan,
                    wat ons $t = 0$ en $t = \frac{4}{3}$ s geeft.
            \end{description}
    \end{enumerate}



\vspace{1cm}
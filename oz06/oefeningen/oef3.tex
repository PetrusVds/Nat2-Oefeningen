\textbf{\underline{OZ 6 - Magnetische inductie en de wet van Faraday - Oefening 3:}}
\vspace{0.5cm}

    \begin{minipage}{.76\textwidth}
        Een homogeen magnetisch veld wordt aangelegd in de cirkel. Het veld verandert in de tijd volgens $B = (2.00t3 - 4.00t2 +0.800) $ T met $t$ de tijd in seconden. Zij $r_2 = 2R = 5.00$ cm.

        \begin{enumerate}[(a)]
            \item Bereken de grootte en de richting van de kracht die inwerkt op een elektron dat zich in een punt $P_2$ bevindt als $t = 2.00$ s.
            \item Op welk tijdstip is deze kracht gelijk aan $0$?
        \end{enumerate}    
    \end{minipage}
    \hspace{0.5cm}\begin{minipage}{.2\textwidth}
        \includegraphics[scale = 0.22]{oz06/resources/Oz6Oef3.png}
    \end{minipage}

    \begin{enumerate}[(a)]
        \item     
            \begin{description}[labelwidth=1.5cm, leftmargin=!]
                \item[Geg. :]
                \item[Gevr. :] 
                \item[Opl. :]
            \end{description}
        \item     
            \begin{description}[labelwidth=1.5cm, leftmargin=!]
                \item[Geg. :]
                \item[Gevr. :] 
                \item[Opl. :]
            \end{description}
    \end{enumerate}



\vspace{1cm}
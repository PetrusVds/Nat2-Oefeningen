\textbf{\underline{OZ 3 - De Lorentzkracht en de wet van Ampère - Oefening 3:}}
\vspace{0.5cm}

% \begin{description}[labelwidth=1.5cm, leftmargin=!]
%     \item[Geg. :]   
%     \item[Gevr. :]  
%     \item[Opl. :]  
% \end{description}

Een zeer lange geleidende strip van breedte $d$ en verwaarloosbare dikte ligt in een horizontaal vlak en draagt een uniforme stroom $I$ door zijn cross sectie
\begin{enumerate}[(a)]
    \item 
        Toon aan dat voor punten op een afstand $y$ recht boven het centrum het magneetveld gegeven is door:
        \begin{equation*}
            B = \dfrac{\mu_0I}{\pi d}\tan^{-1}\left(\dfrac{d}{2y}\right)
        \end{equation*}
        Neem hierbij aan dat de strip oneindig lang is.
    \item 
        Weke waarde benadert $B$ voor $y >> d$? Houdt dit steek? Verklaar.
\end{enumerate}

\begin{enumerate}[(a)]
    \item 
        \begin{description}[labelwidth=1.5cm, leftmargin=!]
            \item[Geg. :]   $d$, $I$, $y$
            \item[Gevr. :]  $B$ ?
            \item[Opl. :]  
                        We kunnen een strip zien als een hoop infinitesimale draden over een breedte $d$. We zullen een
                        infinitesimale magnetisch veld bekijken dat veroorzaakt wordt door één van deze infinitesimale draden. Het $B_y$ veld zal nul worden, we bekijken dus $B_x$.
                        \begin{equation*}
                            dB_x 
                                = \frac{\mu_o}{2\pi}\frac{\sin(\theta)}{r} dI
                                = \frac{\mu_o I}{2\pi d}\frac{\sin(\theta)}{r} dx
                                = \frac{\mu_o I}{2\pi d}\frac{y}{x^2+y^2} dx
                        \end{equation*}
                        Het punt $y$ bevindt zich boven de oorsprong, we gebruiken symmetrie en integreren dus van $0\to\tfrac{d}{2}$: 
                        \begin{equation*}
                            B 
                                = \bigintsss_{\hspace{0.05cm} 0}^{\tfrac{d}{2}} \dfrac{\mu_oIy}{\pi d}\dfrac{dx}{x^2 + y^2} 
                                = \dfrac{\mu_oIy}{\pi d} \bigintsss_{\hspace{0.05cm} 0}^{\tfrac{d}{2}} \dfrac{dx}{x^2 + y^2}
                                = \dfrac{\mu_oI}{\pi d}\tan^{-1}(\tfrac{d}{2y})
                        \end{equation*}
        \end{description}
    \item 
        $\lim_{y\to\infty} \tan^{-1}(\tfrac{d}{2y}) = 0 \Rightarrow B = 0$
\end{enumerate}

\vspace{1cm}
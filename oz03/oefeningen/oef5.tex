\textbf{\underline{OZ 3 - De Lorentzkracht en de wet van Ampère - Oefening 5:}}
\vspace{0.5cm}

Een solenoïde met diameter $10,0$ cm en lengte $75,0$ cm wordt gemaakt van een koperen
draad met een heel dunne laag isolatie. De diameter van de draad is $0,100$ cm.
De draad wordt in een enkele laag rond een kartonnen cilinder gewikkeld, waarbij
opeenvolgende wikkelingen elkaar raken. Welk vermogen moet geleverd worden aan
deze solenoïde om een veld van $8,00$ mT te creëeren in het centrum? De resistiviteit
van koper is $1.75 \cdot 10^{-8} \ \Omega$m bij $20^\circ$.

% \begin{description}[labelwidth=1.5cm, leftmargin=!]
%     \item[Geg. :]   
%     \item[Gevr. :]  
%     \item[Opl. :]  
% \end{description}

\begin{description}[labelwidth=1.5cm, leftmargin=!]
    \item[Geg. :]   $d_{\text{spoel}} = 10.0$ cm, $\ell_{\text{spoel}} = 75.0$ cm, $d_{\text{draad}} = 0.100$ cm, $\rho = 1.75 \cdot 10^{-8} \ \Omega$m, $B = 8.00$ mT
    % $T = 20^\circ$
    \item[Gevr. :]  $P$ ?
    \item[Opl. :]   
    
                    We passen de wet van ampère toe 
                    \begin{equation*}
                        \oint \Vec{B} \cdot d\Vec{\ell} = B\ell_{\text{spoel}} = \mu_0NI_{\text{in}}
                    \end{equation*}
                    Er is gegeven dat opeenvolgende wikkelingen elkaar raken, hieruit volgt dat 
                    \begin{equation*}
                        B = \mu_0\dfrac{N}{\ell_{\text{spoel}}}I_{\text{in}} = \mu_0d_{\text{draad}}I_{\text{in}} 
                    \end{equation*}
                    sinds $N = \tfrac{\ell_{\text{spoel}}}{d_{\text{draad}}}$. We vervangen $I_{\text{in}}$ met $\sqrt{\tfrac{P}{R}}$:
                    \begin{equation*}
                        B = \mu_0\sqrt{\dfrac{P}{R}} = \mu_0\sqrt{\dfrac{P}{4\rho\dfrac{\ell_{\text{draad}}}{\pi d_{\text{draad}}^2}}}
                    \end{equation*}
                    De lengte van de draad kunnen we berekenen met
                    \begin{equation*}
                        \ell_{\text{draad}} = \pi d_{\text{spoel}} N 
                    \end{equation*}
                    Hieruit kunnen we P berekenen: 
                    \begin{equation*}
                        P = \left(\dfrac{B}{\mu_0}\right)^2\left(4\rho N \dfrac{d_{\text{spoel}}}{
                        d_{\text{draad}}^2}\right) \approx 213 W
                    \end{equation*}           
\end{description}

\vspace{1cm}
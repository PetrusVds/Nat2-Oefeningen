% \textbf{\underline{OZ 7 - Inductantie - Oefening 3:}}
% \vspace{0.5cm}

% Door een lange rechte draad met straal $R$ loopt een stroom $I$, uniform verdeeld
% over zijn doorsnede. Bepaal de magnetische energie per lengte-eenheid die wordt
% opgeslagen in het binnenste van de draad.

% \begin{description}[labelwidth=1.5cm, leftmargin=!]
%     \item[Geg. :]  
%     \item[Gevr. :] 
%     \item[Opl. :] (Oplossing van Giancoli 4E Chapter 30 Problem 21)

%     We create an Amperean loop of radius $r$ to calculate the magnetic field within the wire using Eq. 28-3. Since the resulting magnetic field only depends on radius, we use Eq. 30-7 for the energy density in the differential volume $d V=2 \pi r \ell d r$ and integrate from zero to the radius of the wire.
% $$
% \begin{aligned}
% & \oint \overrightarrow{\mathbf{B}} \cdot d \vec{\ell}=\mu_0 I_{e n c} \rightarrow B(2 \pi r)=\mu_0\left(\frac{I}{\pi R^2}\right)\left(\pi r^2\right) \rightarrow B=\frac{\mu_0 I r}{2 \pi R^2} \\
% & \frac{U}{\ell}=\frac{1}{\ell} \int u_B d V=\int_0^R \frac{1}{2 \mu_0}\left(\frac{\mu_0 I r}{2 \pi R^2}\right)^2 2 \pi r d r=\frac{\mu_0 I^2}{4 \pi R^4} \int_0^R r^3 d r=\frac{\mu_0 I^2}{16 \pi}
% \end{aligned}
% $$

    
% \end{description}

% \vspace{1cm}
\textbf{\underline{OZ 10 - De vergelijkingen van Maxwell - Oefening 1:}}
\vspace{0.5cm}

Een zeer lange, dunne staaf draagt elektrische lading met een dichtheid $35.0 \ \text{nC/m}$. De staaf ligt georiënteerd volgens de x-as en beweegt langs de x-as met een snelheid van $15,0 \ \text{Mm/s}$.

\begin{enumerate}[(a)]
    \item Bepaal het elektrisch veld dat de staaf creëert in het punt $P = (0; y; 0)$ met $y = 20.0 \ \text{cm}$.
    \item Bepaal het gecreëerd magnetisch veld in datzelfde punt.
    \item Bepaal de kracht uitgeoefend op een elektron in dat punt, dat beweegt met een snelheid $(240\hat{i})$ Mm/s.
\end{enumerate}

\begin{description}[labelwidth=1.5cm, leftmargin=!]
    \item[Geg. :] $\lambda = 35.0 \ \text{nC/m}$, $v = v_x = 15,0 \ \text{Mm/s}$
\end{description}

\begin{enumerate}[(a)]
    \item 
        \begin{description}[labelwidth=1.5cm, leftmargin=!]
            \item[Geg. :]   $P = (0; y; 0)$ met $y = 20.0 \ \text{cm}$
            \item[Gevr. :]  $\vec{E}_P$
            \item[Opl. :]   
                Het elektrisch veld nabij een lange, rechte draad is gegeven door:
                \begin{equation*}
                    \vec{E}_P = 2k_e\frac{\lambda}{y} \ (\hat{j}) \approx 3.15 \cdot 10^{3} \ (\hat{j}) \ \text{N/C}
                \end{equation*}
        \end{description}
    \item
        \begin{description}[labelwidth=1.5cm, leftmargin=!]
            \item[Geg. :]   $P = (0; y; 0)$ met $y = 20.0 \ \text{cm}$
            \item[Gevr. :] $B_P$
            \item[Opl. :]
                We kunnen de elektrostatische lading op de bewegende draad beschouwen als een stroom:
                \begin{equation*}
                    I =  \lambda v = 0.525 \ \text{A}
                \end{equation*}
                Het magnetisch veld nabij een lange, rechte draad is gegeven door:
                \begin{equation*}
                    B_P = \frac{\mu_0}{2\pi} \frac{I}{y} \ (\hat{k}) \approx 5.25 \cdot 10^{-7} \ (\hat{k}) \ \text{T}
                \end{equation*}
        \end{description}
    \item
        \begin{description}[labelwidth=1.5cm, leftmargin=!]
            \item[Geg. :]  $\vec{v}_e = (240\hat{i})$ Mm/s
            \item[Gevr. :] $F_L$
            \item[Opl. :]   
                De Lorentzkracht op een elektron is gegeven door
                \begin{equation*}
                    \vec{F}_L = q_e(\vec{E} + \vec{v}_e \times \vec{B}) = q_e(E - v_eB) \ (\hat{j}) \approx 4.84 \cdot 10^{-16} \ (-\hat{j}) \ \text{N}
                \end{equation*}
                waarbij de magnetische kracht de elektron afstoot van de staaf en de elektrische kracht de elektron aantrekt.
        \end{description}
\end{enumerate}



\vspace{1cm}
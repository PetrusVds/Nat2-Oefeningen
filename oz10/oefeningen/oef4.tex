\textbf{\underline{OZ 10 - De vergelijkingen van Maxwell - Oefening 4:}}
\vspace{0.5cm}

Een zeer grote parallelle platen condensator draagt een lading met een uniforme lading per eenheid oppervlakte $+\sigma$ op de bovenste plaat en $-\sigma$ op de onderste plaat. Beide platen liggen horizontaal en bewegen horizontaal met een snelheid $v$ naar rechts.

\begin{enumerate}[(a)]
    \item Wat is het magnetisch veld tussen de platen?
    \item Wat is het magnetisch veld in de buurt van de platen, maar buiten de condensator
    \item Wat is de grootte en de richting van de magnetische kracht per eenheid oppervlakte op de bovenste plaat?
    \item Op welke snelheid $v$ zal de magnetische kracht op een plaat gelijk zijn aan de elektrische kracht op die plaat?
\end{enumerate}

\begin{description}[labelwidth=1.5cm, leftmargin=!]
    \item[Geg. :]  $\sigma_+ = +\sigma$, $\sigma_- = -\sigma$, $v$
\end{description}

\begin{enumerate}[(a)]
    \item 
        \begin{description}[labelwidth=1.5cm, leftmargin=!]
            \item[Gevr. :] $B_{\text{in}}$ ?
            \item[Opl. :]   
                Het magnetisch veld door de bovenste plaat vinden we met 
                \begin{equation*}
                    B = \frac{\mu_o J_s}{2} = \frac{\mu_0}{2}\frac{I}{w} = \frac{\mu_0}{2}\sigma v
                \end{equation*}
                met $w$ de zijde waardoor de lading stroomt. Het magnetisch veld door de onderste plaat vinden we met \begin{equation*}
                    B = -\frac{\mu_o J_s}{2} = -\frac{\mu_0}{2}\frac{I}{w} = -\frac{\mu_0}{2}\sigma v.
                \end{equation*}
                Deze velden zullen elkaar versterken, dus het totale magnetisch veld tussen de platen is
                \begin{equation*}
                    \vec{B}_{\text{in}} = \mu_0\sigma v \ (-\hat{k}).
                \end{equation*}
        \end{description}
    \item 
        \begin{description}[labelwidth=1.5cm, leftmargin=!]
            \item[Gevr. :] $B_{\text{uit}}$ ?
            \item[Opl. :]   
                Het zijn oneindige platen (wat we als oneindig veel draden kunnen beschouwen) dus het magnetisch veld zal buiten de platen nul zijn, sinds ze elkaar opheffen.
        \end{description}
    \item 
        \begin{description}[labelwidth=1.5cm, leftmargin=!]
            \item[Gevr. :] $\frac{\vec{F_b}}{A}$ ?
            \item[Opl. :]  
                We berekenen de magnetische kracht op een plaat
                \begin{equation*}
                    \vec{F_b} = I\vec{\ell} \times \vec{B} = I \ell B \ (\hat{j}) = \frac{\mu_0\sigma^2v^2A}{2} \ (\hat{j})
                \end{equation*}
                met $\vec{\ell}$ de lengte van de plaat (in het vlak van het magnetisch veld). De kracht per eenheid oppervlakte is dus
                \begin{equation*}
                    \frac{\vec{F_b}}{A} = \frac{\mu_0\sigma^2v^2}{2} \ (\hat{j}).
                \end{equation*}
        \end{description}
    \item 
        \begin{description}[labelwidth=1.5cm, leftmargin=!]
            \item[Gevr. :] $v \Rightarrow F_e = F_b$ ?
            \item[Opl. :] 
                De elektrische kracht op een plaat is
                \begin{equation*}
                    \vec{F_e} = q \vec{E} = \sigma A \frac{\sigma}{2\epsilon_0} \ (-\hat{j}) = A \frac{\sigma^2}{2\epsilon_0} \ (-\hat{j}).
                \end{equation*}
                We stellen de krachten aan elkaar gelijk en vinden voor $v$:
                \begin{equation*}
                    v = \sqrt{\frac{1}{\epsilon_0\mu_0}} \approx c
                \end{equation*}
        \end{description}
\end{enumerate}

% \begin{description}[labelwidth=1.5cm, leftmargin=!]
%     \item[Geg. :]   
%     \item[Gevr. :] 
%     \item[Opl. :]   
% \end{description}

\vspace{1cm}
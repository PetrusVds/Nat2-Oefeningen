% \textbf{\underline{OZ 9 - Wisselstroomkringen - Oefening 2:}}
% \vspace{0.5cm}

% Een weerstand van $ 20,0 \ \Omega $ en een spoel met inductantie $ L = 25,0 $ mH worden in serie geplaatst met een wisselspanningsbron met frequentie 60,0 Hz en een rms spanning van 120 V.

% \begin{enumerate}[(a)]
%     \item Wat is de rms stroom die door het circuit gaat lopen?
%     \item Wat is de fasehoek tussen de spanning en de stroom?
%     \item Welke capaciteit moet er aan het circuit worden toegevoegd om de fasehoek 0 te maken?
%     \item Welke rms spanning moet men instellen op de spanningsbron om ervoor te zorgen dat het vermogen dat de spanningsbron levert aan dit $ RLC $-circuit gelijk is aan het vermogen dat geleverd werd aan het $ RL $-circuit?
% \end{enumerate}

% \begin{description}[labelwidth=1.5cm, leftmargin=!]
%     \item[Geg. :]   $ R = 20,0 \ \Omega $; $ L = 25,0 $ mH; $ \Delta V_{rms} = 120 $ V; $ f = 60,0 $ Hz;
% \end{description}

% \begin{enumerate}[(a)]
%     \item
%         \begin{description}[labelwidth=1.5cm, leftmargin=!]
%             \item[Gevr. :]  $ I_{rms} $;
%             \item[Opl. :]   $ X_L = \omega L 
%                             = 2 \pi f L 
%                             = 2 \pi \cdot 60,0 \cdot 25,0 \cdot 10^-3 
%                             = 3 \pi \ \Omega $
                            
%                             $ Z = \sqrt{R^2 + Z_L^2} = \sqrt{20,0^2 + \left( 3 \pi \right)^2} = 22,1094197 \ \Omega $
                            
%                             $ I_{rms} = \dfrac{\Delta V_{rms}}{Z} 
%                             = \dfrac{120}{22,1094197} 
%                             = 5,427550863 $ A $ 
%                             \approx 5,43 $ A
%         \end{description}
%     \item
%         \begin{description}[labelwidth=1.5cm, leftmargin=!]
%             \item[Gevr. :]  $ \phi $;
%             \item[Opl. :]   $ \phi = \tan^{-1}{\dfrac{X_L}{R}}
%                             = \tan^{-1}{\dfrac{3 \pi}{20,0}} 
%                             = 25,2316372^{\circ} 
%                             \approx 25,2^{\circ} $
%         \end{description}
%     \item
%         \begin{description}[labelwidth=1.5cm, leftmargin=!]
%             \item[Gevr. :]  $ C $ zodat $ \phi = 0 $;
%             \item[Opl. :]   $ \phi = 0 \to X_C = X_L $
            
%                             \hspace{-0.57cm} $ \Leftrightarrow 
%                             \dfrac{1}{\omega C} = \omega L $
            
%                             \hspace{-0.57cm} $ \Leftrightarrow 
%                             \dfrac{1}{2 \pi f C} = 2 \pi f L $
            
%                             \hspace{-0.57cm} $ \Leftrightarrow 
%                             C = \dfrac{1}{4 \pi^2 f^2 L}
%                             = \dfrac{1}{4 \pi^2 \cdot 60,0^2 \cdot 25,0 \cdot 10^{-3}} 
%                             = 0,000281448 $ F $
%                             \approx 281 \ \mu $F 
%         \end{description}
%     \item
%         \begin{description}[labelwidth=1.5cm, leftmargin=!]
%             \item[Gevr. :]  $ \Delta V_{rms} $ zodat $ P_{RL,av} = P_{RLC,av} $;
%             \item[Opl. :]   $ P_{RL,av} = I_{RL,rms}^2 R 
%                             = 5,43^2 \cdot 20,0 
%                             = 589,698 $ W
                            
%                             \vspace{0.5cm}
                            
%                             $ X_C = X_L \to Z = R \to P_{RLC,av} = I_{RLC,rms} \cdot V_{RLC,rms} = \dfrac{V_{RLC,rms}^2}{R} $
                            
%                             \hspace{-0.57cm} $ \Leftrightarrow 
%                             V_{RLC,rms} = \sqrt{P_{RLC,av} R} 
%                             = \sqrt{589,698 \cdot 20,0} 
%                             = 108,6 $ V $ 
%                             \approx 109 $ V
%         \end{description}
% \end{enumerate}

% \vspace{1cm}
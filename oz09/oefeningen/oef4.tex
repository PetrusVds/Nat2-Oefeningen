\textbf{\underline{OZ 9 - Wisselstroomkringen - Oefening 4:}}
\vspace{0.5cm}

In de schakeling in onderstaande figuur, neem aan dat alle parameters behalve C gegeven zijn.

\begin{enumerate}[(a)]
    \item Bepaal de stromen in functie van de tijd indien beide schakelaars gesloten zijn.
    \item Vind het vermogen geleverd aan de schakeling. \vspace{0.3cm} \\ 
\begin{minipage}{.56\textwidth}
    \item Vind de stroom in functie van de tijd na het openen van enkel schakelaar $S_1$.   
    \item Nadat schakelaar $S_2$ ook geopend is, zijn de stroom en de spanning in fase. Vind de capaciteit $C$.
    \item Vind de impedantie van het circuit wanneer beide schakelaars geopend zijn bij deze frequentie.
    \item Vind de maximale energie opgeslagen in de condensator en de spoel tijdens de oscillaties.
    \item De frequentie van de spanningsbron wordt nu verdubbeld. Vind het faseverschil tussen de stroom en de spanning.
    \item Vind de frequentie van de bronspanning die ervoor zorgt dat de inductieve reactantie de helft is van de capacitieve reactantie.
\end{minipage}
\begin{minipage}{.4\textwidth}
   \vspace{-0.5cm}\includegraphics[scale = 0.4]{oz09/resources/Oz9Oef4.png}
\end{minipage}

\end{enumerate}

\begin{enumerate}[(a)]
    \item 
        \begin{description}[labelwidth=1.5cm, leftmargin=!]
            \item[Geg. :]   
            \item[Gevr. :] 
            \item[Opl. :]   
        \end{description}
    \item
        \begin{description}[labelwidth=1.5cm, leftmargin=!]
            \item[Geg. :]   
            \item[Gevr. :] 
            \item[Opl. :]   
        \end{description}
    \item
        \begin{description}[labelwidth=1.5cm, leftmargin=!]
            \item[Geg. :]   
            \item[Gevr. :] 
            \item[Opl. :]   
        \end{description}
    \item
        \begin{description}[labelwidth=1.5cm, leftmargin=!]
            \item[Geg. :] Resonantie !!! 
            \item[Gevr. :] 
            \item[Opl. :]   
        \end{description}
    \item
        \begin{description}[labelwidth=1.5cm, leftmargin=!]
            \item[Geg. :]   
            \item[Gevr. :] 
            \item[Opl. :]   
        \end{description}
        \item
        \begin{description}[labelwidth=1.5cm, leftmargin=!]
            \item[Geg. :]   
            \item[Gevr. :] 
            \item[Opl. :]   
        \end{description}
    \item
        \begin{description}[labelwidth=1.5cm, leftmargin=!]
            \item[Geg. :]   
            \item[Gevr. :] 
            \item[Opl. :]   
        \end{description}
    \item
        \begin{description}[labelwidth=1.5cm, leftmargin=!]
            \item[Geg. :]   
            \item[Gevr. :] 
            \item[Opl. :]   
        \end{description}
\end{enumerate}

% \begin{description}[labelwidth=1.5cm, leftmargin=!]
%     \item[Geg. :]   
%     \item[Gevr. :] 
%     \item[Opl. :]   
% \end{description}

\vspace{1cm}
% \textbf{\underline{OZ 9 - Wisselstroomkringen - Oefening 3:}}
% \vspace{0.5cm}

% Beschouw een seriële $ RLC $-schakeling die gevoed wordt door een wisselspanningsbron met een vaste frequentie en een vaste rms spanning, waarbij de weerstand $ R $ gelijk is aan de inductieve reactantie. Als de afstand tussen de platen van de condensator gehalveerd wordt, verdubbelt de stroom in het circuit. Bepaal de initiële capacitieve reactantie in functie van $ R $.

% \begin{description}[labelwidth=1.5cm, leftmargin=!]
%     \item[Geg. :]   $ \Delta V $; $ f $; $ d_e = \dfrac{d}{2} $; $ I_e = 2 I $;
%     \item[Gevr. :]  $ X_C(R) $;
%     \item[Opl. :]   Als $ d_e = \dfrac{d}{2} $ dan $ C_e = 2 C $ want $ C = \dfrac{\varepsilon_0 A}{d} $ 
    
%                     Als $ C_e = 2 C $ dan $ X_{C,e} = \dfrac{X_C}{2} $ want $ X_C = \dfrac{1}{\omega C} $ 
    
%                     Als $ C_e = 2 C $ dan $ X_{C,e} = \dfrac{X_C}{2} $ want $ X_C = \dfrac{1}{\omega C} $ 
                    
%                     Als $ I_e = 2 I $ dan $ Z_e = \dfrac{1}{2} Z $ want $ Z = \dfrac{\Delta V}{I} $ 
                    
%                     $ Z_e = \dfrac{Z}{2} $
                    
%                     \hspace{-0.57cm} $ \Leftrightarrow 
%                     \sqrt{R^2 + \left( X_L - X_{C,e} \right)^2} = \dfrac{\sqrt{R^2 + \left( X_L - X_C \right)^2}}{2} $
                    
%                     \hspace{-0.57cm} $ \Leftrightarrow 
%                     R^2 + \left( X_L - X_{C,e} \right)^2 = \dfrac{R^2 + \left( X_L - X_C \right)^2}{4} $
                    
%                     \hspace{-0.57cm} $ \Leftrightarrow 
%                     R^2 + \left( R - \dfrac{X_C}{2} \right)^2 = \dfrac{R^2 + \left( R - X_C \right)^2}{4} $
                    
%                     \hspace{-0.57cm} $ \Leftrightarrow 
%                     R^2 + R^2 - R X_C - \dfrac{X_C^2}{4} = \dfrac{R^2 + R^2 - 2R X_C + X_C^2}{4} $
                    
%                     \hspace{-0.57cm} $ \Leftrightarrow 
%                     2R^2 - R X_C - \dfrac{X_C^2}{4} = \dfrac{R^2}{2} - \dfrac{R X_C}{2} + \dfrac{X_C^2}{4} $
                    
%                     \hspace{-0.57cm} $ \Leftrightarrow 
%                     \left( R  \dfrac{R}{2} \right) X_C = 2R^2 - \dfrac{R^2}{2} $
                    
%                     \hspace{-0.57cm} $ \Leftrightarrow 
%                     \dfrac{1 R}{2}  X_C = \dfrac{3 R^2}{2} $
                    
%                     \hspace{-0.57cm} $ \Leftrightarrow 
%                     X_C = 3R $
% \end{description}

% \vspace{1cm}